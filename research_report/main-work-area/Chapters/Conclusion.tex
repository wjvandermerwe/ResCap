\chapter{Conclusion}
\label{Chapter4}


\noindent In this study, I explored the application of Random Survival Forests (RSF) and Cox Proportional Hazards models in survival analysis, focusing on evaluating their performance through dynamic parameter tuning and the use of synthetic data generated by Survival GAN and Survival VAE models. The simulation results demonstrated that both models provide valuable insights into the underlying patterns of the dataset, with the RSF model showing slightly improved performance in survival prediction and cumulative hazard estimates. However, the Cox model, once convergence issues were addressed through preprocessing and variable selection techniques, proved to be a robust alternative for simpler survival data structures.
\\\\
\noindent The comparison between the models revealed that while the Cox model performed well in terms of metrics like concordance and calibration, the RSF model's flexibility in handling complex interactions and high-dimensional data provided it with an edge, particularly in Brier score and integrated Brier score (IBS). Despite the computational cost associated with tuning parameters in the RSF model, the results highlight its effectiveness in survival analysis, especially when dealing with more intricate datasets.
\\\\
\noindent In addition, the exploratory data analysis (EDA) provided crucial insights into the structure of the dataset, with the correlation matrix and bivariate analysis identifying key relationships between variables, such as the skewed distribution of age and its effect on survival. The visualizations generated through the RSF and Cox models helped validate these findings, further emphasizing the importance of selecting appropriate models based on the complexity of the data.
\\\\
\noindent Ultimately, this simulationstudy underscores the value of using a robust framework like ADMEP \parencite{morris_using_2019} to easily apply advanced machine learning techniques like RSF in survival analysis, particularly in scenarios where high-dimensional data and complex variable interactions come into play. The results also suggest that model selection should be guided by both performance metrics and the computational feasibility of tuning parameters, especially in resource-constrained environments.
\\\\
\noindent The metrics employed, such as concordance, Brier score, integrated Brier score (IBS), and calibration errors, are critical for evaluating the performance of survival models across diverse datasets. By applying these metrics consistently, a fair comparison can be drawn between the Random Survival Forest (RSF) and Cox Proportional Hazards (CoxPH) models, providing a comprehensive understanding of how each model performs under various data structures \parencite{harrell__regression_2015}. This approach ensures that model evaluation is not solely based on a single dataset, which might lead to biased conclusions. Instead, it allows for generalizability across multiple datasets, capturing a wider range of complexities and ensuring that model performance is robust and adaptable to different survival scenarios \parencite{tibshirani_regression_1996}. The dynamic nature of the metrics, especially those like Brier score and IBS, further aids in understanding model behavior over time, making them more relevant for long-term predictions in survival analysis \parencite{qi_survivaleval_2024}.
\\\\
\noindent Moreover, the use of simulation models like Survival GAN and Survival VAE not only facilitates ethical clearance by circumventing the need for sensitive or limited real-world data but also allows for the introduction of controlled variance and different survival patterns \parencite{norcliffe_survivalgan_2023}. This flexibility in generating synthetic data is invaluable in survival analysis, as it enables the testing of models under various conditions, including varying degrees of censoring, time-dependent covariates, and event rates. By introducing such controlled variance, simulations provide a more rigorous testing ground for survival models, ensuring that they can handle diverse data types and scenarios before applying them to real-world datasets. This capability enhances the models' adaptability and resilience, ultimately leading to more reliable and generalizable insights across different survival analysis applications \parencite{qian_synthcity_2023}.