\chapter{Appendix Title} % Main appendix title
\label{AppendixA} % For referencing this appendix elsewhere, use \ref{AppendixA}

\section{Main Section}

\noindent Lorem ipsum dolor sit amet, consectetur adipiscing elit. Aliquam ultricies lacinia euismod. Nam tempus risus in dolor rhoncus in interdum enim tincidunt. Donec vel nunc neque. In condimentum ullamcorper quam non consequat. Fusce sagittis tempor feugiat. Fusce magna erat, molestie eu convallis ut, tempus sed arcu.

\par\vspace{0.5cm}
\noindent \Cshadowbox{
	\begin{minipage}{15cm}
		\bigskip
		\color[rgb]{0.0,0.4,0.65} The appendices are sections in which complicated mathematical or other formulae, descriptions of experiments or apparatus, and any other specialised or lengthy material such as computer programme listings, copies of spectra or other instrumental outputs are found.  
		\medskip
\end{minipage}}\\

\begin{table}[h!]
    \centering
    \begin{tabular}{|l|l|p{10cm}|}
    \hline
    \textbf{Label} & \textbf{Category} & \textbf{Description} \\ \hline
    DP   & Data Preparation (General)   & General label for all data preparation tasks, including cleaning, transformation, and preprocessing. \\ \hline
    MDH  & Missing Data Handling        & Specific tasks related to identifying, handling, and imputing missing data. \\ \hline
    ENC  & Data Encoding                & Tasks related to encoding categorical variables, feature scaling, and other data transformations. \\ \hline
    DS   & Data Splitting               & Splitting the data into training, validation, and testing sets. \\ \hline
    MI   & Model Initialization         & Initializing models with chosen parameters, including setting up cross-validation schemes. \\ \hline
    MT   & Model Training (General)     & General label for tasks related to model training, including fitting the model to training data. \\ \hline
    HT   & Hyperparameter Tuning        & Tasks focused on optimizing model parameters, such as grid search or random search. \\ \hline
    REG  & Regularization Application   & Applying regularization techniques like Lasso, Ridge, or Elastic Net during model fitting. \\ \hline
    SA   & Survival Analysis (General)  & General label for survival analysis tasks, including survival function estimation and survival curve plotting. \\ \hline
    CIA  & Covariate Impact Analysis    & Analyzing the impact of individual covariates on the outcome, including statistical tests and visualizations. \\ \hline
    ME   & Model Evaluation (General)   & General label for evaluating model performance across various metrics. \\ \hline
    CV   & Cross-Validation             & Performing cross-validation to assess model performance stability. \\ \hline
    AC   & Assumption Checking (General) & General label for checking model assumptions, including proportional hazards or other key assumptions. \\ \hline
    BVT  & Bias-Variance Tradeoff       & Analyzing the tradeoff between bias and variance in model performance. \\ \hline
    SS   & Simulation Setup             & Setting up the simulation environment, including configuring parameters and preparing input data. \\ \hline
    SDG  & Synthetic Data Generation (General) & General label for generating synthetic datasets using trained models. \\ \hline
    SDE  & Synthetic Data Evaluation (General) & General label for evaluating the quality and utility of generated synthetic data. \\ \hline
    RC   & Results Compilation          & Compiling and summarizing results from various stages of the study for reporting. \\ \hline
    DR   & Documentation and Reporting  & Documenting processes, decisions, and findings throughout the study for transparency and reproducibility. \\ \hline
    CM   & Comparison of Models         & Comparing results from different models or approaches, such as CoxPH vs. RSF. \\ \hline
    \end{tabular}
    \caption{Labeling Strategy for Simulation Study}
    \label{tab:labeling_strategy}
    \end{table}