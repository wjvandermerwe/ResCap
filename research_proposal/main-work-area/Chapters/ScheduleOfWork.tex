\chapter{Schedule of Work}
\label{Chapter3} % For referencing this chapter elsewhere, use \ref{Chapter4}

\section{Schedule of Work}

\noindent Here a rough estimate of timelines are scheduled, evenly spaced to fit the total available timeline, this is due to the synchronously coupled nature of the tasks. It should be noted that tasks that get completed speedily should increase available time for the other tasks. An exmaple could be that I stumble on a thourough publicly available dataset that include preprocessing steps spesifically for survival analysis which speeds up the task 1.1 and 1.2 thereby allowing more freedom to experiment, and verify data validity against the other steps.
\\\\
\noindent \begin{tabularx}{\textwidth}{|>{\hsize=0.5\hsize}X|>{\hsize=1.5\hsize}X|>{\hsize=0.5\hsize}X|>{\hsize=1.5\hsize}X|}
	\hline
	\textbf{Objective} & \textbf{Title} & \textbf{Task} & \textbf{Description} \\
	\hline

	1 & \multirow{2}{=}{\hsize=1.0\hsize Sourcing a Dataset} & 1.1 & Initial Data Collection \\
	\cline{3-4} 
	& & 1.2 & Data Cleaning and Preprocessing \\
	\hline

	2 & \multirow{2}{=}{\hsize=1.0\hsize Applying Data Generating Mechanism} & 2.1 & Develop Simulation Models \\
	\cline{3-4}
	& & 2.2 & Generate Synthetic Data \\
	\hline

	3 & \multirow{2}{=}{\hsize=1.0\hsize Execute Survival Models} & 3.1 & Model Implementation \\
	\cline{3-4}
	& & 3.2 & Model Optimization \\
	\hline

	4 & \multirow{2}{=}{\hsize=1.0\hsize Perform Evaluation Metrics and Analysis} & 4.1 & Metrics Calculation \\
	\cline{3-4}
	& & 4.2 & Results Analysis \\
	\hline

	5 & \multirow{2}{=}{\hsize=1.0\hsize Formulate Report} & 5.1 & Results Visualisation \\
	\cline{3-4}
	& & 5.2 & Qualitative Interpretations and report writing \\
	\hline
\end{tabularx}

\medskip

\newgeometry{left=1cm, right=1cm, top=1cm, bottom=1cm}
\begin{landscape}
\thispagestyle{empty}
\vspace*{\fill} % Add vertical space before your content to center it
\begin{center}
\begin{ganttchart}[
	time slot format=isodate,
	time slot unit=day,
	x unit=0.36cm,
	y unit title=0.5cm,
	y unit chart=0.7cm,
	hgrid,
	vgrid,
	group right shift=0,
	group top shift=.6,
	group height=.3,
	group peaks tip position=0,
	group peaks height=.2,
	group peaks width=.2,
	title label font=\scriptsize,
	bar label font=\scriptsize,
	milestone label font=\scriptsize,
	]{2024-05-20}{2024-07-09}
	\gantttitlecalendar{year, month=name, day} \\
	
	\ganttgroup{Objective 1}{2024-05-28}{2024-06-17} \\
	\ganttbar{Task 1.1}{2024-05-28}{2024-06-06} \\
	\ganttbar{Task 1.2}{2024-06-07}{2024-06-17} \\
	
	\ganttgroup{Objective 2}{2024-06-18}{2024-07-08} \\
	\ganttbar{Task 2.1}{2024-06-18}{2024-06-27} \\
	\ganttbar{Task 2.2}{2024-06-28}{2024-07-08} \\

\end{ganttchart}
\end{center}
\vspace*{\fill}
\end{landscape}
\restoregeometry
\medskip

\newgeometry{left=1cm, right=1cm, top=1cm, bottom=1cm}
\begin{landscape}
\thispagestyle{empty}
\vspace*{\fill} % Add vertical space before your content to center it
\begin{center}
\begin{ganttchart}[
	time slot format=isodate,
	time slot unit=day,
	x unit=0.35cm,
	y unit title=0.5cm,
	y unit chart=0.7cm,
	hgrid,
	vgrid,
	group right shift=0,
	group top shift=.6,
	group height=.3,
	group peaks tip position=0,
	group peaks height=.2,
	group peaks width=.2,
	title label font=\scriptsize,
	bar label font=\scriptsize,
	milestone label font=\scriptsize,
	]{2024-07-08}{2024-09-14}
	\gantttitlecalendar{year, month=name, day} \\
	
	\ganttgroup{Objective 3}{2024-07-09}{2024-07-29} \\
	\ganttbar{Task 3.1}{2024-07-09}{2024-07-18} \\
	\ganttbar{Task 3.2}{2024-07-19}{2024-07-29} \\
	
	\ganttgroup{Objective 4}{2024-07-30}{2024-08-19} \\
	\ganttbar{Task 4.1}{2024-07-30}{2024-08-08} \\
	\ganttbar{Task 4.2}{2024-08-09}{2024-08-19} \\

	\ganttgroup{Objective 5}{2024-08-20}{2024-09-08} \\
	\ganttbar{Task 5.1}{2024-08-20}{2024-08-23} \\
	\ganttbar{Task 5.2}{2024-08-24}{2024-09-08} \\
	
\end{ganttchart}
\end{center}
\vspace*{\fill}
\end{landscape}
\restoregeometry

\medskip

\section{Potential Difficulties}


\begin{itemize}
	\item The complexity of setting up, executing, and analysing multiple simulation scenarios to rigorously compare these two methodologies demands meticulous time management to meet academic requirements and achieve comprehensive results.
	\item Furthermore as stated in \ref{methlim} it is quite important to be mindfull of deadlines for the task components, due to the tight coupling that could affect overarching success.
	\item Lastly, the success of my research hinges on my ability to accurately implement and analyse the comparative study using specialised statistical software. Due to the complex nature of some of the fundamental proofs there are many considerations to account for successful implementation.
\end{itemize}


