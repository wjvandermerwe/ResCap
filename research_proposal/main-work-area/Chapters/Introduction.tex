\chapter{Introduction} % Main chapter title
\label{Chapter1} % For referencing this chapter elsewhere, use \ref{Chapter1}

\noindent test

\noindent Lorem ipsum dolor sit amet, consectetur adipiscing elit. Aliquam ultricies lacinia euismod. Nam tempus risus in dolor rhoncus in interdum enim tincidunt. Donec vel nunc neque. In condimentum ullamcorper quam non consequat. Fusce sagittis tempor feugiat. Fusce magna erat, molestie eu convallis ut, tempus sed arcu. 

\par\vspace{0.5cm}
\noindent \Cshadowbox{
	\begin{minipage}{15cm}
		\bigskip
		\color[rgb]{0.0,0.4,0.65} The introductory chapter introduces and motivates the proposed research. It should provide the reader with enough information to understand what the proposed research is, how it relates to its broader context, and why it is worthwhile. \par \medskip	
		Throughout the proposal, you should write as if your reader has knowledge of your field broadly, but no more specific or domain knowledge than this. You should try to include as much information as a reader like this would need to understand the research (and no more than this). \par \medskip
		The introduction to the chapter itself should
		\begin{itemize}
			\item introduce the broad area of research/problem area,
			\item introduce the more specific area of research/problem area, 
			\item introduce the research question or problem.
		\end{itemize}
		\medskip
\end{minipage}}

\section{Literature review}

\noindent 




\par\vspace{0.5cm}
\noindent \Cshadowbox{
	\begin{minipage}{15cm}
		\bigskip
		\color[rgb]{0.0,0.4,0.65} The role of a literature review section (or chapter) is, first, to show your familiarity with the research relevant to your proposed research and, secondly, to motivate the proposed research (that is, to provide support for the claim that the research question has not yet been asked or answered or has not yet been adequately asked or answered). \par \medskip		
		The literature review section should 
		\begin{itemize}
			\item begin with a description of the current state of research on the proposed research question,
			\item provide a systematic survey of the literature relevant to the proposed research that motivates it,
			\item conclude by indicating how the proposed research is motivated in light of the survey above.
		\end{itemize}
		 A literature review can be structured in a number of different ways (for example, chronologically, thematically, methodologically). You should choose a structure suitable to motivating your proposed research.
		
		\medskip
\end{minipage}} 

\subsection{Subsection} 

\noindent Lorem ipsum dolor sit amet, consectetur adipiscing elit. Aliquam ultricies lacinia euismod. Nam tempus risus in dolor rhoncus in interdum enim tincidunt. Donec vel nunc neque.

\section{Problem Statement}

\noindent Morbi rutrum odio eget arcu adipiscing sodales. Aenean et purus a est pulvinar pellentesque.Cras in elit neque,  quis varius elit.   Phasellus fringilla,  nibh eu tempus venenatis,  dolor elitposuere quam, quis adipiscing urna leo nec orci.  Sed nec nulla auctor odio aliquet consequat.Ut nec nulla in ante ullamcorper aliquam at sed dolor.  Phasellus fermentum magna in auguegravida cursus. Cras sed pretium lorem. Pellentesque eget ornare odio. Proin accumsan, massaviverra cursus pharetra, ipsum nisi lobortis velit, a malesuada dolor lorem eu neque. 

\par\vspace{0.5cm}
\noindent \Cshadowbox{
	\begin{minipage}{15cm}
		\bigskip
		\color[rgb]{0.0,0.4,0.65} If your research involves identifying a problem, this section will put forward a statement of the problem. A problem statement typically begins with a description of the ideal situation (relevant to the research), describes the current real situation (relevant to the research), and finally states the way in which the proposed
		research will bring the current situation closer to the described ideal situation. If your research does not involve identifying a problem, this section can be omitted.
		 \par	
		\medskip
\end{minipage}} 

\section{Research Question}

\noindent Morbi rutrum odio eget arcu adipiscing sodales. Aenean et purus a est pulvinar pellentesque.Cras in elit neque,  quis varius elit.   Phasellus fringilla,  nibh eu tempus venenatis,  dolor elitposuere quam, quis adipiscing urna leo nec orci.  Sed nec nulla auctor odio aliquet consequat.Ut nec nulla in ante ullamcorper aliquam at sed dolor.

\par\vspace{0.5cm}
\noindent \Cshadowbox{
	\begin{minipage}{15cm}
		\bigskip
		\color[rgb]{0.0,0.4,0.65} This section of the chapter states and discusses in a clear and succinct way the question that the research will seek to answer. The question should be precise, unambiguous, succinct, and sufficiently narrow. If your research involves hypothesis-led methods, this section identifies and discusses the hypothesis of
		the research.
		\par	
		\medskip
\end{minipage}} 


\section{Research Aims and Objectives}

Lorem ipsum dolor sit amet, consectetur adipiscing elit. Aliquam ultricies lacinia euismod. Nam tempus risus in dolor rhoncus in interdum enim tincidunt. Donec vel nunc neque. In condimentum ullamcorper quam non consequat. Fusce sagittis tempor feugiat. Fusce magna erat, molestie eu convallis ut, tempus sed arcu. 

\subsection{Research Aims}

Sed ullamcorper quam eu nisl interdum at interdum enim egestas. Aliquam placerat justo sed lectus lobortis ut porta nisl porttitor. Vestibulum mi dolor, lacinia molestie gravida at, tempus vitae ligula. 

\par\vspace{0.5cm}
\noindent \Cshadowbox{
	\begin{minipage}{15cm}
		\bigskip
		\color[rgb]{0.0,0.4,0.65} This section of the chapter identifies the aim of the research. The ’aim’ of the research is the overarching and broadest goal of the research. Remember that it must be a goal that the research literally aims to achieve. For example, if the problem of the research is a lack of accurate classification models for a certain
		type of image within some domain, the aim of the research cannot be to ’develop a model that will be widely adopted within the domain’. This is something that you as the researcher might hope for, but it is not literally a goal of the research. The (literal) aim of the research would be ’to develop a model for the classification
		of images of type X...’. Remember that the aim is stated using the infinitive. For example, ’the aim of the research is \emph{to develop} a model...’
		
		\par	
		\medskip
\end{minipage}} 


\newpage
\subsection{Objectives}

Sed ullamcorper quam eu nisl interdum at interdum enim egestas. Aliquam placerat justo sed lectus lobortis ut porta nisl porttitor. Vestibulum mi dolor, lacinia molestie gravida at, tempus vitae ligula. Donec eget quam sapien, in viverra eros.

\par\vspace{0.5cm}
\noindent \Cshadowbox{
	\begin{minipage}{15cm}
		\medskip
		\color[rgb]{0.0,0.4,0.65} This section of the chapter identifies the objectives of the research. The 'objectives' of the research are the smaller tasks that are carried out in order to achive your aim. You should identify 4 to 6 objectives. You should ensure that your objectives are precise, measurable, achievable, and directly linked to your research aim. Again, remember that objectives are stated using the infinitive: 'The objectives of the research are \emph{to construct} a data set... etc.’
		\medskip 
\end{minipage}}\\ 
 
\section{Limitations}

Sed ullamcorper quam eu nisl interdum at interdum enim egestas. Aliquam placerat justo sed lectus lobortis ut porta nisl porttitor. Vestibulum mi dolor, lacinia molestie gravida at, tempus vitae ligula. Donec eget quam sapien, in viverra eros.

\par\vspace{0.5cm}
\noindent \Cshadowbox{
	\begin{minipage}{15cm}
		\medskip
		\color[rgb]{0.0,0.4,0.65}This section of the chapter addresses the scope of the research by identifying and discussing its limitations. These limitations might have a number of sources. For example, the size and extent of the data set used for the research could limit its generalisability. In this section, you should identify and discuss any such limitations that are foreseeable at the time of proposal.
		\medskip 
\end{minipage}}\\

\section{Assumptions and Definitions}

Morbi rutrum odio eget arcu adipiscing sodales. Aenean et purus a est pulvinar pellentesque.Cras in elit neque,  quis varius elit. Phasellus fringilla,  nibh eu tempus venenatis,  dolor elitposuere quam, quis adipiscing urna leo nec orci.  

\par\vspace{0.5cm}
\noindent \Cshadowbox{
	\begin{minipage}{15cm}
		\bigskip
		\color[rgb]{0.0,0.4,0.65} Like the section above, this section addresses the scope of the research, along with providing any relevant technical definitions from the literature. It should include the following.
			\begin{description}
				\item[Assumptions] Identify and discuss any notable assumptions of the research.
				\item[Definitions] Define any important technical terms appearing in the research.
			\end{description}
		\medskip 
\end{minipage}}\\

\section{Overview}

Lorem ipsum dolor sit amet, consectetur adipiscing elit. Aliquam ultricies lacinia euismod. Nam tempus risus in dolor rhoncus in interdum enim tincidunt. Donec vel nunc neque. In condimentum ullamcorper quam non consequat. Fusce sagittis tempor feugiat. 

\par\vspace{0.5cm}
\noindent \Cshadowbox{
	\begin{minipage}{15cm}
		\bigskip
		{\color[rgb]{0.0,0.4,0.65}This brief and final section provides an overview of the proposal that follows. Remember to keep this section brief and to-the-point.}
		\medskip 
\end{minipage}}\\
