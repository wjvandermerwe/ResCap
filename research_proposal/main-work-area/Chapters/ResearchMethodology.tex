\chapter{Research Methodology}
\label{Chapter2} % For referencing this chapter elsewhere, use \ref{Chapter3}

\noindent Lorem ipsum dolor sit amet, consectetur adipiscing elit. Aliquam ultricies lacinia euismod. Nam tempus risus in dolor rhoncus in interdum enim tincidunt. Donec vel nunc neque. In condimentum ullamcorper quam non consequat. Fusce sagittis tempor feugiat.
\par\vspace{0.5cm}
\noindent \Cshadowbox{
    \begin{minipage}{15cm}
    	\bigskip
    	\color[rgb]{0.0,0.4,0.65}The methodology chapter provides an overview and description of all the important elements of \emph{how} the research will be carried out. It differs from the methodology chapter of the final report insofar as it includes discussion 
    		\begin{itemize}
    			\item both of available and chosen methods, and,
    			\item of any foreseeable limitations of the methods. 
    		\end{itemize}
    	\medskip
    \end{minipage}}\\

\section{Research design}

\noindent Morbi rutrum odio eget arcu adipiscing sodales. Aenean et purus a est pulvinar pellentesque.Cras in elit neque,  quis varius elit.   Phasellus fringilla,  nibh eu tempus venenatis,  dolor elitposuere quam, quis adipiscing urna leo nec orci.  
\par Lorem ipsum dolor sit amet, consectetur adipiscing elit. Aliquam ultricies lacinia euismod. Nam tempus risus in dolor rhoncus in interdum enim tincidunt. Donec vel nunc neque. In condimentum ullamcorper quam non consequat. Fusce sagittis tempor feugiat. Fusce magna erat, molestie eu convallis ut, tempus sed arcu. Quisque molestie, ante a tincidunt ullamcorper, sapien enim dignissim lacus, in semper nibh erat lobortis purus. 

\par\vspace{0.5cm}
\noindent \Cshadowbox{
	\begin{minipage}{15cm}
		\bigskip
		\color[rgb]{0.0,0.4,0.65}This section provides a brief, high-level description of the broad research method to be used in carrying out the research. This research method will vary from field to field. The section should 
			\begin{itemize}
				\item identify the broad research method, and,
				\item provide a brief description of the method.
		\end{itemize} 
		\medskip
\end{minipage}}\\

%\section{Data}

%\noindent Sed ullamcorper quam eu nisl interdum at interdum enim egestas. Aliquam placerat justo sed lectus lobortis ut porta nisl porttitor. Vestibulum mi dolor, lacinia molestie gravida at, tempus vitae ligula. Donec eget quam sapien, in viverra eros.

%\par\vspace{0.5cm}
%\noindent \Cshadowbox{
%	\begin{minipage}{15cm}
%		\bigskip
%		\color[rgb]{0.0,0.4,0.65}This section deals with the data set(s) to be used in the research. It provides a description of these and discusses the pre-processing of the data set(s) that will be carried out for the research. It should include the following.
%			\begin{description}
%				\item[Source] Identify and describe the source of the data set(s).
%				\item[Data] Identify and describe the data set(s) themselves (features, records, etc.).
%				\item[Pre-processing] Identify and describe the pre-processing steps to be taken. 
%		\end{description} 
%		If your research does not involve data, this section can be omitted.
		\medskip
%end{minipage}}\\

\section{Methods}

\noindent Lorem ipsum dolor sit amet, consectetur adipiscing elit. Aliquam ultricies lacinia euismod. Nam tempus risus in dolor rhoncus in interdum enim tincidunt. Donec vel nunc neque. In condimentum ullamcorper quam non consequat. Fusce sagittis tempor feugiat. 

\par\vspace{0.4cm}
\noindent \Cshadowbox{
	\begin{minipage}{15cm}
		\medskip
		\color[rgb]{0.0,0.4,0.65}This section describes and \emph{motivates} the instruments and procedure to be used in carrying out the research. These should not be discussed in a chronological way (e.g. first, this step will be taken, and then this step will be taken, etc.), but should instead be grouped together systematically. A more detailed discussion of the relevant algorithms or models should be included here. The discussion should show an understanding of the \emph(available) methods and put forward a \emph{motivation for those chosen}. 
		
		%The section should include the following.
		%	\begin{description}
		%		\item[Instruments] Identify, e.g., software, architecture, algorithms, models, etc.
		%		\item[Decisions] Identify, e.g., training/test split, class imbalance, feature selection, etc.
		%		\item[Procedure] Identify, e.g., number of runs, prevention of data leakage, etc.
		%		\item[Data] Describe the data that will be generated by the methods.
		%\end{description} 
		\medskip
\end{minipage}}\\

%\section{Analysis}

%Morbi rutrum odio eget arcu adipiscing sodales. Aenean et purus a est pulvinar pellentesque.Cras in elit neque,  quis varius elit.   Phasellus fringilla,  nibh eu tempus venenatis,  dolor elitposuere quam, quis adipiscing urna leo nec orci.

%\par\vspace{0.4cm}
%\noindent \Cshadowbox{
%	\begin{minipage}{15cm}
%		\medskip
%		\color[rgb]{0.0,0.4,0.65}This section describes the analysis of the data that will be generated in carrying out the research. It should describe how you will determine the significance of your results. The section should include the following.
%			\begin{description}
%				\item[Descriptive Statistics] Identify, e.g., Mean, Standard Deviations, etc.
%				\item[Metrics] Identify, e.g., Mean Average Precision, Recall, etc.
%				\item[Baselines] Identify the relevant baselines in the literature.
%				\item[Comparisons] Identify, e.g., confusion matrices, AUC-ROC curve, etc. 
%		\end{description}
%		\medskip
%\end{minipage}}\\

\section{Limitations}

\noindent Sed ullamcorper quam eu nisl interdum at interdum enim egestas. Aliquam placerat justo sed lectus lobortis ut porta nisl porttitor. Vestibulum mi dolor, lacinia molestie gravida at, tempus vitae ligula. Donec eget quam sapien, in viverra eros. 

\par\vspace{0.4cm}
\noindent \Cshadowbox{
	\begin{minipage}{15cm}
		\medskip
		\color[rgb]{0.0,0.4,0.65}This section identifies limitations with the methodology. These limitations might have a number of sources. For example, constraints on time and/or computational power could limit the number of methods or models tested. In this section, you should identify and discuss any such limitations that are foreseeable at the time of proposal.
		\medskip
\end{minipage}}\\

\section{Ethical Considerations}

\noindent Lorem ipsum dolor sit amet, consectetur adipiscing elit. Aliquam ultricies lacinia euismod. Nam tempus risus in dolor rhoncus in interdum enim tincidunt. Donec vel nunc neque. In condimentum ullamcorper quam non consequat. Fusce sagittis tempor feugiat. Fusce magna erat, molestie eu convallis ut, tempus sed arcu. Quisque molestie, ante a tincidunt ullamcorper, sapien enim dignissim lacus, in semper nibh erat lobortis purus. Integer dapibus ligula ac risus convallis pellentesque.

\par\vspace{0.4cm}
\noindent \Cshadowbox{
	\begin{minipage}{15cm}
		\medskip
		\color[rgb]{0.0,0.4,0.65}This section assesses the need for ethical clearance for the proposed research.In cases in which no ethical clearance is required, the section will indicate as such and include a brief motivation for this. In cases in which ethical clearance is required, the section will indicate as such, include a brief motivation for this, and describe the procedure that will be followed for obtaining clearance.
		\medskip
\end{minipage}}\\

